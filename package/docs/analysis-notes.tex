% description of analysis options in jTRACE
% $Id: analysis-notes.tex 198 2005-07-13 23:05:38Z harlan $

\documentclass{article}
\usepackage{amsmath}
\usepackage{amssymb}
\usepackage{fullpage}

\def\argmax{\mathop{\rm arg\,max}}

\begin{document}

\title{Analysis Options in jTRACE}

\maketitle

\section{Introduction}

This document describes, in mathematical detail, the analysis options in
the Graphing... Analysis... tab of jTRACE. 

Words or Phonemes can be analyzed. Recall that for each cycle of the model,
both of these matricies is two-dimensional, where one dimension is the item
(particular word or phoneme) and the other dimension is a time slice, 
indicating the model's representation of past, present, and future. 

The content of an analysis can be either raw Activations, or Luce Choice
Rule (LCR) Response Probabilities. If the latter, then the Luce Choice options
(described below) can be specified.

A subset of Items can be displayed. If Response Probabilities are used, 
probabilities are computed for only those words/phonemes.

The Alignment can be specified, where Alignment refers to which of the time
slices is used. When Alignment is Specified, then only that time slice is
used. For example, when preceded by the silence phone, '-', jTRACE words 
are generally activated at time slice 4, so setting Specified = 4 when
graphing words will result
in a graph of words that are aligned with the start of the utterance. If 
phonemes are being graphed, or
parameters are changed, or more or less silence is used, or multiple words
are present, other options may make more sense. Average alignment averages
over all time slices, for each item. Max (Ad-Hoc) finds the best alignment for
each item, for each time cycle. Max (Post-Hoc) finds the single
best alignment for each word. The Frauenfelder rule is the average activation
of the item at the specified and subsequent alignment, where all items that
have time slices overlapping with the target item are potential responses.
(This rule is a bit clearer mathematically than in words.)

\section{Notation}

$R$ is a 3-D matrix of the raw data, $R_{c,i,s}$, where $c \in [0..C-1]$ is
time cycle, and $C$ is the total number of cycles, $i \in [0..I-1]$ is
the item index, and $I$ is the total number of items, and $s \in [0..S-1]$ is
the time slice, and $S$ is the total number of time slices. For example,
$R_{2,3,4}$ is the raw activation of the $i$'th item aligned with time slice
$s$, after $c$ steps of the model.

$R'$ is the response strength, calculated as $R' = e^{(\frac{R-min}{max-min})
k}$, where $min$ and $max$ are the minimum and maximum node activation values,
and $k \in \mathbb{N}$ is the usual exponent parameter in the LCR.

$P$ is the 2-D matrix of plottable data, the output of the analysis process,
$P_{i,c}$, where $i \in [0..\vert w \vert-1]$, and $\vert w \vert$ is the
number of specified items, and $c$ is the time cycle, as above.

$w \subset {0,I-1}$ is a set of indicies, selected by the user, to be plot,
and indexed $w_j$.

$a \in {0..S-1}$ is the specified alignment.

\section{Analysis options}

There are 3 possible analyses (Activations, Response Probabilities
with Normal Choice, and Response Probabilities with Forced Choice), and
5 possible alignments (Average, Max (Ad-Hoc), Max (Post-Hoc), Specified, and
Frauenfelder), giving 15 possible computations.

\vspace{.3in}

%\begin{small}
\begin{tabular}{r|p{1.7in}|p{1.7in}|p{1.7in}}
 & Activations/* & RPs/Normal & RPs/Forced \\
\hline
Average &
\begin{minipage}{1.7in}
\begin{gather*}
\forall j, 0 \le j < \vert w \vert, \forall c, 0 \le c < C,\\
P_{j,c} = \frac{1}{S} \sum_{s=0}^{S-1} R_{c,w_j,s}.
\end{gather*}
\end{minipage}
&
\begin{minipage}{1.7in}
\begin{gather*}
\forall j, 0 \le j < \vert w \vert, \forall c, 0 \le c < C,\\
P_{j,c} = \frac{1}{S} \sum_{s=0}^{S-1} 
\frac{R'_{c,w_j,s}}{\sum_{i=0}^{I-1} R'_{c,i,s}}.
\end{gather*}
\end{minipage}
& 
\begin{minipage}{1.7in}
\begin{gather*}
\forall j, 0 \le j < \vert w \vert, \forall c, 0 \le c < C,\\
P_{j,c} = \frac{1}{S} \sum_{s=0}^{S-1} 
\frac{R'_{c,w_j,s}}{\sum_{k=0}^{\vert w \vert} R'_{c,w_k,s}}.
\end{gather*}
\end{minipage}
\\
\hline
Max (Ad-Hoc) &
\begin{minipage}{1.7in}
\begin{gather*}
\forall i, 0 \le i < I, \forall c, 0 \le c < C,\\
a_{i,c} = \argmax_a R_{c,i,a}, \\
\forall j, 0 \le j < \vert w \vert, \forall c, 0 \le c < C,\\
P_{j,c} = R_{c,w_j,a_{w_j,c}}.
\end{gather*}
\end{minipage}
&
\begin{minipage}{1.7in}
\begin{gather*}
\forall i, 0 \le i < I, \forall c, 0 \le c < C,\\
a_{i,c} = \argmax_a R_{c,i,a}, \\
\forall j, 0 \le j < \vert w \vert, \forall c, 0 \le c < C,\\
P_{j,c} = \frac{R'_{c,w_j,a_{w_j,c}}}
{\sum_{\iota=0}^{I-1} R'_{c,\iota,a_{w_\iota,c}}}.
\end{gather*}
\end{minipage}
& 
\begin{minipage}{1.7in}
\begin{gather*}
\forall i, 0 \le i < I, \forall c, 0 \le c < C,\\
a_{i,c} = \argmax_a R_{c,i,a}, \\
\forall j, 0 \le j < \vert w \vert, \forall c, 0 \le c < C,\\
P_{j,c} = \frac{R'_{c,w_j,a_{w_j,c}}}
{\sum_{k=0}^{\vert w \vert} R'_{c,w_k,a_{w_k,c}}}.
\end{gather*}
\end{minipage}
\\
\hline
Max (Post-Hoc) &
\begin{minipage}{1.7in}
\begin{gather*}
\forall i, 0 \le i < I \\
a_{i} = \argmax_a \max_c R_{c,i,a}, \\
\forall j, 0 \le j < \vert w \vert, \forall c, 0 \le c < C,\\
P_{j,c} = R_{c,w_j,a_{w_j}}.
\end{gather*}
\end{minipage}
&
\begin{minipage}{1.7in}
\begin{gather*}
\forall i, 0 \le i < I \\
a_{i} = \argmax_a \max_c R_{c,i,a}, \\
\forall j, 0 \le j < \vert w \vert, \forall c, 0 \le c < C,\\
P_{j,c} = \frac{R'_{c,w_j,a_{w_j}}}
{\sum_{i=0}^{I-1} R'_{c,i,a_{w_j}}}.
\end{gather*}
\end{minipage}
& 
\begin{minipage}{1.7in}
\begin{gather*}
\forall i, 0 \le i < I \\
a_{i} = \argmax_a \max_c R_{c,i,a}, \\
\forall j, 0 \le j < \vert w \vert, \forall c, 0 \le c < C,\\
P_{j,c} = \frac{R'_{c,w_j,a_{w_j}}}
{\sum_{k=0}^{\vert w \vert} R'_{c,w_k,a_{w_k}}}.
\end{gather*}
\end{minipage}
\\
\hline
Specified &
\begin{minipage}{1.7in}
\begin{gather*}
\forall j, 0 \le j < \vert w \vert, \forall c, 0 \le c < C,\\
P_{j,c} = R_{c,w_j,a}.
\end{gather*}
\end{minipage}
&
\begin{minipage}{1.7in}
\begin{gather*}
\forall j, 0 \le j < \vert w \vert, \forall c, 0 \le c < C,\\
P_{j,c} = \frac{R'_{c,w_j,a}}
{\sum_{i=0}^{I-1} R'_{c,i,a}}.
\end{gather*}
\end{minipage}
& 
\begin{minipage}{1.7in}
\begin{gather*}
\forall j, 0 \le j < \vert w \vert, \forall c, 0 \le c < C,\\
P_{j,c} = \frac{R'_{c,w_j,a}}
{\sum_{k=0}^{\vert w \vert} R'_{c,w_k,a}}.
\end{gather*}
\end{minipage}
\\
\hline
Frauenfelder &
\begin{minipage}{1.7in}
\begin{gather*}
\forall j, 0 \le j < \vert w \vert, \forall c, 0 \le c < C,\\
P_{j,c} = \frac{1}{2} (R_{c,w_j,a} + R_{c,w_j,a+1}).
\end{gather*}
\end{minipage}
&
\begin{minipage}{1.7in}
\begin{gather*}
\forall j, 0 \le j < \vert w \vert, \forall c, 0 \le c < C,\\
P_{j,c} = \frac{(R_{c,w_j,a} + R_{c,w_j,a+1})} \\
{ 2 \sum_{s=0}^{S-1} \sum_{i=0}^{I-1} R'_{c,i,s} \Omega_{s}}, \text{where}\\
\Omega_{s} = 1 \text{if} a - z < s < a + z + t^{length}, \\
z = w_i^{length} \frac{\text{deltaInput}}{\text{slicesPerPhon}}, \\
w_i^{length} \text{is the length of item} i,\\
t^{length} \text{is the length of the target item}.
\end{gather*}
\end{minipage}
& 
3
\end{tabular}
%\end{small}

\end{document}


